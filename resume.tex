%-------------------------
% Resume in Latex
% Author : ChatGPT based on your HTML
%-------------------------

\documentclass[a4paper,11pt]{article}

\usepackage{kotex} % 한글 사용
\usepackage[margin=1in]{geometry}
\usepackage{titlesec}
\usepackage{enumitem}
\usepackage{hyperref}
\usepackage{parskip}
\usepackage{xcolor}

% Section formatting
\titleformat{\section}{\large\bfseries}{}{0em}{}[\titlerule]

% Custom commands
\newcommand{\tech}[1]{\textbf{\textcolor{blue}{#1}}}

\begin{document}

\begin{center}
    {\LARGE \textbf{경력 기술서}} \\[6pt]
    Email: \href{mailto:kty991213@naver.com}{kty991213@naver.com} \quad
    Mobile: (+82) 010-4032-4803
\end{center}

%-------------------------------------------
\section{경상남도교육청 미래교육원 (2024.03 ~ Present, 4명)}

\textbf{소속}: ㈜데이타이음  

\textbf{프로젝트 소개}:  
체험 프로그램 예약, 연수 관리, 유료 결제까지 지원하는 통합 플랫폼.  
예약 신청부터 결제, 알림톡 발송, QR코드 기반 현장 참여 확인 및 퀴즈/점수 부여까지  
교육 체험 전 과정을 디지털화하여 운영 효율성과 사용자 편의성을 높임.  

\textbf{기술 스택}: \tech{Java, Spring, MyBatis, JSP, MariaDB, Quartz, TOSS PG, Biztalk}  

\textbf{문제/해결/성과}:
\begin{itemize}[leftmargin=*]
  \item \textbf{문제:} 피크 시간대 동시 예약 처리에서 오버부킹과 데드락 발생.
  \item \textbf{해결:} DB 행 잠금(\texttt{SELECT ... FOR UPDATE}) + 백오프 기반 재시도 로직으로 트랜잭션 안정화. \\
  → 동시 예약 환경에서는 \textbf{비관적 락(Pessimistic Lock)}이 낙관적 락보다 적합하다고 판단해 채택.
  \item \textbf{성과:} 오버부킹 0건, 동시 200명 예약 성공률 100\% 유지, 예약 UX와 결제 안정성 개선.
\end{itemize}

%-------------------------------------------
\section{경남교육청 진로교육원 봄봄 플랫폼 (2025.08 ~ Present, 3명)}

\textbf{소속}: ㈜데이타이음  

\textbf{프로젝트 소개}:  
체험 프로그램 예약·결제·만족도 조사까지 지원하는 통합 플랫폼.  
사용자 단: 특별 프로그램 예약, 설문, 비회원 휴대폰 인증 예약, 길찾기.  
관리자 단: 프로그램 관리, 예약 승인/취소, 결제취소, 만족도 통계, 체험실 배정.  

\textbf{기술 스택}: \tech{Java, Spring MVC, MyBatis, JSP, MariaDB, 네이버페이 PG, Biztalk, Quartz, 카카오맵 API, 사이렌 휴대폰 인증}  

\textbf{문제/해결/성과}:
\begin{itemize}[leftmargin=*]
  \item \textbf{문제:} 피크 시간대 동시 예약 시 오버부킹, 결제 중복, 승인 지연 가능성.
  \item \textbf{해결:} DB 행 잠금 + 백오프 기반 재시도 로직으로 동시성 제어. 네이버페이 콜백 검증·트랜잭션 분리로 이중결제 차단. 예약 승인/취소·환불 절차 표준화. \\
  → 교육청 체험예약 특성상 \textbf{동시성·결제 안정성}을 최우선으로 설계.
  \item \textbf{성과:} 동시 150명 예약 성공률 100\% 검증, 결제 중복 0건, 길찾기/인증 기능으로 사용자 편의성 향상.
\end{itemize}

%-------------------------------------------
\section{충북 통합 도서관 웹사이트 (2024.01 ~ Present, 3명)}

\textbf{소속}: ㈜데이타이음  

\textbf{프로젝트 소개}:  
도내 20개 교육도서관을 하나로 묶어 예약·대출·전자도서관 서비스를 제공하는 통합 사이트.  
사용자는 한 계정으로 모든 도서관과 체험 예약 서비스를 이용 가능.  

\textbf{기술 스택}: \tech{Java, Spring MVC, iBatis, JSP, OracleDB}  

\textbf{문제/해결/성과}:
\begin{itemize}[leftmargin=*]
  \item \textbf{문제:} 일정이 목록 형태로만 제공돼 예약 가능 여부 확인이 불편.
  \item \textbf{해결:} 캘린더 UI 도입, 인덱스 추가 및 중복 조회 최소화로 DB 성능 개선.
  \item \textbf{성과:} 예약 조회 속도 수초 → 1초 이내 단축, 월 1,000건 예약 안정적 처리, 취소/변경 처리 속도 30\% 단축.
\end{itemize}

%-------------------------------------------
\section{충청북도교육청 다채움 플랫폼 (2024.01 ~ 2025.09, 9명)}

\textbf{소속}: ㈜데이타이음  

\textbf{프로젝트 소개}:  
AI·빅데이터 기반 학습 지원 플랫폼.  
학습 이력·포트폴리오 관리, 맞춤형 학습, 수업 공유, 독서 관리, 설문/피드백 기능 제공.  

\textbf{기술 스택}: \tech{Java, Spring, MyBatis, JSP, MariaDB/Oracle/MySQL, REST API, Swagger, Bcrypt, Firebase Cloud Messaging}  

\textbf{문제/해결/성과}:
\begin{itemize}[leftmargin=*]
  \item \textbf{문제:} Firebase SDK 인증 오류로 모바일 푸시 실패.
  \item \textbf{해결:} 서비스 계정 키 적용, \texttt{FirebaseApp.initializeApp()} 단일 초기화, 문서화로 재발 방지.
  \item \textbf{성과:} 알림 실패율 15\% → 0\% 개선, 설문 API 안정화, 실시간 알림 동기화로 신뢰도 회복.
\end{itemize}

%-------------------------------------------
\section{알콩 플랫폼 (2025.07 ~ Present, 6명)}

\textbf{소속}: ㈜데이타이음  

\textbf{프로젝트 소개}:  
AI 기반 학습 통합 플랫폼, SCAP 보안인증 대응 중.  

\textbf{기술 스택}: \tech{Java, Spring, MyBatis, JSP, MariaDB/Oracle/MySQL, 보안 스캐너}  

\textbf{주요 성과}:
\begin{itemize}[leftmargin=*]
  \item CODE-RAY 기반 취약점 진단·패치 반복, 시큐어코딩 적용 → 취약점 0건 유지.
  \item 보안 조치 표준화로 패치·검증 속도 40\% 단축, 감리 대응 신뢰성 확보.
\end{itemize}

\end{document}
