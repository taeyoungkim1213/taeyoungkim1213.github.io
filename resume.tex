%-------------------------
% Resume in LaTeX
%-------------------------

\documentclass[a4paper,11pt]{article}

\usepackage{kotex} % 한글 사용
\usepackage[margin=1in]{geometry}
\usepackage{titlesec}
\usepackage{enumitem}
\usepackage{hyperref}
\usepackage{parskip}
\usepackage{xcolor}

% Section formatting
\titleformat{\section}{\large\bfseries}{}{0em}{}[\titlerule]

% Custom commands
\newcommand{\tech}[1]{\textbf{\textcolor{blue}{#1}}}

\begin{document}

\begin{center}
    {\LARGE \textbf{김태영 경력 기술서}} \\[6pt]
    Email: \href{mailto:kty991213@naver.com}{kty991213@naver.com} \quad
    Mobile: (+82) 010-4032-4803
\end{center}

\vspace{0.5cm}

\noindent \tech{Java}, \tech{Spring MVC}, \tech{MyBatis/iBatis}, \tech{MariaDB/Oracle/MySQL}, \tech{트랜잭션/동시성}, \tech{보안/SSO}

%-------------------------------------------
\section{충청북도교육청 다채움 플랫폼 (2024.01 ~ 2025.09, 9명 참여)}

\textbf{소속}: ㈜데이타이음 \\
\textbf{프로젝트 소개}:  
충청북도교육청이 추진한 ‘다채움’은 초·중·고 학생의 학습 이력, 포트폴리오,  
비인지적 요소(자기조절학습 등)를 통합 관리하고 AI·빅데이터 기반으로 학습 수준 진단과  
맞춤형 학습을 지원하는 학생 성장 지원 플랫폼.  
교사·학생·학부모 모두가 참여하는 통합 서비스로 수업 설계 공유, 형성평가, 독서 활동 관리,  
전자책 서비스, 설문 및 피드백 기능 등을 제공하며 현재 교육청 및 산하기관에서 실제 운영 중.

\textbf{주요 업무}:
\begin{itemize}[leftmargin=*]
  \item 관리자 페이지에서 회원 관리, 설문조사 관리, 시스템 알림, 게시판(FAQ·공지사항) 기능 개발
  \item 회원가입 시 개인정보 암호화 및 비밀번호 초기화 기능 구현 → 보안 강화
  \item 클래스 모듈 내 알림 기능 개발 및 \tech{Firebase Cloud Messaging(FCM)} 연동 → 모바일 푸시 알림 지원
  \item 통합회원 모듈에 객관식·주관식·파일 업로드형 설문 기능 개발, API 형태로 외부 기관 제공
  \item 시스템 전반 유지보수 및 고도화 참여
\end{itemize}

\textbf{문제}: 모바일 푸시 연동 중 Firebase SDK 인증 설정 오류로 알림 전송이 실패함. \\
\textbf{해결}: 서비스 계정 키(JSON)를 적용하고 \verb|FirebaseApp.initializeApp()|을 단일 초기화하도록 로직 개선. 초기화 과정을 문서화해 이후 개발/운영 환경에서도 동일하게 적용. \\
\textbf{성과}:
\begin{itemize}[leftmargin=*]
  \item SDK 초기화 오류 해결 → 웹·모바일 간 실시간 알림 제공
  \item 알림 누락·지연 최소화 → 교사·학생 간 소통 효율성 강화
  \item 외부 서비스 연동 안정화 경험 확보
\end{itemize}

\textbf{기술 스택}: \tech{Java, Spring, MyBatis, JSP, MariaDB, Oracle, MySQL, REST API, Swagger, Bcrypt, Firebase Cloud Messaging}

%-------------------------------------------
\section{경상남도교육청 미래교육원 (2024.03 ~ , 4명 참여)}

\textbf{소속}: ㈜데이타이음 \\
\textbf{프로젝트 소개}:  
경상남도교육청이 운영하는 ‘미래교육원’은 학생·교직원·학부모 대상의  
체험 프로그램 예약, 연수 관리, 유료 결제까지 지원하는 통합 플랫폼.  
예약 신청부터 결제, 알림톡 발송, QR코드 기반 현장 참여 확인 및 퀴즈/점수 부여까지  
교육 체험 전 과정을 디지털화하여 운영 효율성과 사용자 편의성을 높임.

\textbf{주요 업무}:
\begin{itemize}[leftmargin=*]
  \item 체험 프로그램 예약/결제 트랜잭션 로직 설계 및 동시성 제어 구현
  \item 결제 모듈을 \tech{TOSS PG}로 연동하고 결제 검증·콜백 처리 강화
  \item 예약 완료 및 취소 시 \tech{비즈톡 알림톡} 발송 기능 개발
  \item QR코드 기반 퀴즈 점수 부여 및 현장 이벤트 처리 기능 개발
  \item \tech{Quartz}를 활용한 휴면 계정 자동 삭제 구현
  \item 시스템 전반 유지보수 및 장애 대응 참여
\end{itemize}

\textbf{문제}: 피크 시간대 동시 예약 처리에서 좌석 초과(오버부킹)와 간헐적 데드락으로 예약 지연 발생. \\
\textbf{해결}: DB 행 잠금(\verb|SELECT ... FOR UPDATE|)을 적용하여 좌석 동시 잠금. 데드락 발생 시 백오프 기반 재시도 로직을 구현해 안정적 트랜잭션 재실행. \\
\textbf{성과}:
\begin{itemize}[leftmargin=*]
  \item 오버부킹 제거 → 동시예약 시 좌석 초과 0건 달성
  \item 데드락 상황에서도 예약 안정적 처리 → 서비스 신뢰도 강화
  \item TOSS 결제 연동 + 비즈톡 알림 도입 → 결제 및 예약 안내 UX 개선
\end{itemize}

\textbf{기술 스택}: \tech{Java, Spring, MyBatis, JSP, MariaDB, Quartz, TOSS PG, Biztalk}

%-------------------------------------------
\section{충북 통합 도서관 웹사이트 (2024.01 ~ , 3명 참여)}

\textbf{소속}: ㈜데이타이음 \\
\textbf{프로젝트 소개}:  
충청북도 내 20개 교육도서관을 통합해  
도서 검색, 예약, 대출, 전자도서관 서비스를 제공하는 통합 사이트.  
사용자는 한 계정으로 모든 도서관 이용 가능하며, 체험 프로그램 예약까지 통합 제공.

\textbf{주요 업무}:
\begin{itemize}[leftmargin=*]
  \item \textbf{일일 도서관 체험 예약 캘린더} 개발 (달력에서 날짜별 프로그램 조회·예약 가능)
  \item 회원·도서 예약/대출 기능 유지보수 및 개선
  \item 느린 SQL 쿼리 수정 및 검색 속도 개선
  \item 운영 중 발생하는 오류 해결 및 기능 추가
\end{itemize}

\textbf{문제}: 체험 프로그램 일정이 목록 형태로만 제공돼 원하는 날짜 예약 가능 여부 확인이 불편함. \\
\textbf{해결}: 캘린더 UI 도입 → 날짜별 프로그램 표시, 달력에서 바로 예약 가능. 검색 인덱스 추가 및 중복 조회 제거로 속도 개선. \\
\textbf{성과}:
\begin{itemize}[leftmargin=*]
  \item 달력 기반 예약 기능 → 사용자 편의성 향상
  \item 월 1,000건 이상의 체험 예약 안정적 처리
  \item 검색 응답 속도 수 초 → 1초 이내로 단축
\end{itemize}

\textbf{기술 스택}: \tech{Java, Spring MVC, iBatis, JSP, OracleDB}

%-------------------------------------------
\section{알콩 플랫폼 (2025.07 ~ , 6명 참여)}

\textbf{소속}: ㈜데이타이음 \\
\textbf{프로젝트 소개}:  
AI 기반 학습 통합 플랫폼으로 학습 관리·콘텐츠·분석 기능을 제공하며,  
현재 \tech{SCAP 보안인증}을 준비 중.

\textbf{주요 업무}:
\begin{itemize}[leftmargin=*]
  \item SCAP(보안인증) 대응을 위한 취약점 점검 및 패치 수행
  \item CODE-RAY 활용 내부 진단 → 취약점 수정 → 재검사 반복
  \item 시큐어 코딩(입력 검증·출력 이스케이프·CSRF 대응) 적용 및 공통 모듈 정리
  \item 감리 대응 및 보안 이행 관리
\end{itemize}

\textbf{성과}:
\begin{itemize}[leftmargin=*]
  \item 정기 스캔·조치로 현재 취약점 0건 유지 (감리 진행 중)
  \item 보안 조치 표준화 → 패치·검증 속도 향상 및 유지보수 부담 경감
\end{itemize}

\textbf{나의 역할}:
\begin{itemize}[leftmargin=*]
  \item 취약점 진단·재현·패치 주도 및 감리 대응 창구 역할
  \item 보안 가이드 작성 및 팀 교육 지원
\end{itemize}

\textbf{기술 스택}: \tech{Java, Spring, MyBatis, JSP, MariaDB, Oracle, MySQL, 보안 스캐너}

\end{document}
